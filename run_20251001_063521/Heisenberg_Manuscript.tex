% arXiv-compatible LaTeX document
% Options for packages loaded elsewhere
\PassOptionsToPackage{hyphens}{url}
\documentclass[
]{article}
\usepackage[T1]{fontenc}
\usepackage[utf8]{inputenc}
\usepackage{textcomp} % provide euro and other symbols
\usepackage{lmodern}
\usepackage{xcolor}
\usepackage{amsmath,amssymb}
\setcounter{secnumdepth}{-\maxdimen} % remove section numbering
% Use upquote if available, for straight quotes in verbatim environments
\IfFileExists{upquote.sty}{\usepackage{upquote}}{}
\IfFileExists{microtype.sty}{% use microtype if available
  \usepackage[]{microtype}
  \UseMicrotypeSet[protrusion]{basicmath} % disable protrusion for tt fonts
}{}
\makeatletter
\@ifundefined{KOMAClassName}{% if non-KOMA class
  \IfFileExists{parskip.sty}{%
    \usepackage{parskip}
  }{% else
    \setlength{\parindent}{0pt}
    \setlength{\parskip}{6pt plus 2pt minus 1pt}}
}{% if KOMA class
  \KOMAoptions{parskip=half}}
\makeatother
\usepackage{color}
\usepackage{fancyvrb}
\newcommand{\VerbBar}{|}
\newcommand{\VERB}{\Verb[commandchars=\\\{\}]}
\DefineVerbatimEnvironment{Highlighting}{Verbatim}{commandchars=\\\{\}}
% Add ',fontsize=\small' for more characters per line
\newenvironment{Shaded}{}{}
\newcommand{\AlertTok}[1]{\textcolor[rgb]{1.00,0.00,0.00}{\textbf{#1}}}
\newcommand{\AnnotationTok}[1]{\textcolor[rgb]{0.38,0.63,0.69}{\textbf{\textit{#1}}}}
\newcommand{\AttributeTok}[1]{\textcolor[rgb]{0.49,0.56,0.16}{#1}}
\newcommand{\BaseNTok}[1]{\textcolor[rgb]{0.25,0.63,0.44}{#1}}
\newcommand{\BuiltInTok}[1]{\textcolor[rgb]{0.00,0.50,0.00}{#1}}
\newcommand{\CharTok}[1]{\textcolor[rgb]{0.25,0.44,0.63}{#1}}
\newcommand{\CommentTok}[1]{\textcolor[rgb]{0.38,0.63,0.69}{\textit{#1}}}
\newcommand{\CommentVarTok}[1]{\textcolor[rgb]{0.38,0.63,0.69}{\textbf{\textit{#1}}}}
\newcommand{\ConstantTok}[1]{\textcolor[rgb]{0.53,0.00,0.00}{#1}}
\newcommand{\ControlFlowTok}[1]{\textcolor[rgb]{0.00,0.44,0.13}{\textbf{#1}}}
\newcommand{\DataTypeTok}[1]{\textcolor[rgb]{0.56,0.13,0.00}{#1}}
\newcommand{\DecValTok}[1]{\textcolor[rgb]{0.25,0.63,0.44}{#1}}
\newcommand{\DocumentationTok}[1]{\textcolor[rgb]{0.73,0.13,0.13}{\textit{#1}}}
\newcommand{\ErrorTok}[1]{\textcolor[rgb]{1.00,0.00,0.00}{\textbf{#1}}}
\newcommand{\ExtensionTok}[1]{#1}
\newcommand{\FloatTok}[1]{\textcolor[rgb]{0.25,0.63,0.44}{#1}}
\newcommand{\FunctionTok}[1]{\textcolor[rgb]{0.02,0.16,0.49}{#1}}
\newcommand{\ImportTok}[1]{\textcolor[rgb]{0.00,0.50,0.00}{\textbf{#1}}}
\newcommand{\InformationTok}[1]{\textcolor[rgb]{0.38,0.63,0.69}{\textbf{\textit{#1}}}}
\newcommand{\KeywordTok}[1]{\textcolor[rgb]{0.00,0.44,0.13}{\textbf{#1}}}
\newcommand{\NormalTok}[1]{#1}
\newcommand{\OperatorTok}[1]{\textcolor[rgb]{0.40,0.40,0.40}{#1}}
\newcommand{\OtherTok}[1]{\textcolor[rgb]{0.00,0.44,0.13}{#1}}
\newcommand{\PreprocessorTok}[1]{\textcolor[rgb]{0.74,0.48,0.00}{#1}}
\newcommand{\RegionMarkerTok}[1]{#1}
\newcommand{\SpecialCharTok}[1]{\textcolor[rgb]{0.25,0.44,0.63}{#1}}
\newcommand{\SpecialStringTok}[1]{\textcolor[rgb]{0.73,0.40,0.53}{#1}}
\newcommand{\StringTok}[1]{\textcolor[rgb]{0.25,0.44,0.63}{#1}}
\newcommand{\VariableTok}[1]{\textcolor[rgb]{0.10,0.09,0.49}{#1}}
\newcommand{\VerbatimStringTok}[1]{\textcolor[rgb]{0.25,0.44,0.63}{#1}}
\newcommand{\WarningTok}[1]{\textcolor[rgb]{0.38,0.63,0.69}{\textbf{\textit{#1}}}}
\usepackage{graphicx}
\makeatletter
% Set default figure placement to htbp
\def\fps@figure{htbp}
% Simple bounded scaling command for arXiv compatibility
\newsavebox\pandoc@box
\newcommand*\pandocbounded[1]{%
  \sbox\pandoc@box{#1}%
  \ifdim\wd\pandoc@box>\linewidth
    \resizebox{\linewidth}{!}{\usebox{\pandoc@box}}%
  \else
    \usebox{\pandoc@box}%
  \fi
}
\makeatother
\setlength{\emergencystretch}{3em} % prevent overfull lines
\providecommand{\tightlist}{%
  \setlength{\itemsep}{0pt}\setlength{\parskip}{0pt}}
\IfFileExists{xurl.sty}{\usepackage{xurl}}{} % add URL line breaks if available
\urlstyle{same}
% arXiv-compatible hyperref setup
\usepackage[pdftex,hidelinks]{hyperref}
\hypersetup{
  pdfcreator={LaTeX via pandoc}}

\author{}
\date{}

\begin{document}

\section{Numerical Demonstration of the Heisenberg Uncertainty Principle
using Gaussian Wave
Packets}\label{numerical-demonstration-of-the-heisenberg-uncertainty-principle-using-gaussian-wave-packets}

\textbf{Stefan Len}

\emph{Independent Researcher}

\textbf{Date:} October 20, 2025

\begin{center}\rule{0.5\linewidth}{0.5pt}\end{center}

\subsection{Abstract}\label{abstract}

I present a minimal numerical demonstration of the Heisenberg
Uncertainty Principle using Gaussian wave packets in one spatial
dimension. By systematically varying the initial position spread, I
numerically verify the reciprocal relationship between position and
momentum uncertainties and confirm that their product at the initial
time achieves the theoretical minimum of ℏ/2. In addition, I simulate
the free time evolution of the wave packet, illustrating the spreading
of the position uncertainty while the momentum distribution remains
constant. The results highlight the role of Gaussian wave packets as
minimum-uncertainty states at the initial moment and provide an
accessible, reproducible teaching tool for quantum mechanics.

\textbf{Keywords:} Heisenberg uncertainty principle, Gaussian wave
packet, quantum mechanics, minimum uncertainty state, numerical
simulation, pedagogical tool

\begin{center}\rule{0.5\linewidth}{0.5pt}\end{center}

\subsection{1. Introduction}\label{introduction}

\subsubsection{1.1 The Heisenberg Uncertainty
Principle}\label{the-heisenberg-uncertainty-principle}

The Heisenberg Uncertainty Principle, formulated by Werner Heisenberg in
1927 {[}1{]}, represents one of the most fundamental departures of
quantum mechanics from classical physics. It establishes that certain
pairs of physical quantities---canonical conjugates such as position and
momentum---cannot be simultaneously measured with arbitrary precision.
Mathematically, for position x and momentum p:

\[\Delta x \cdot \Delta p \geq \frac{\hbar}{2}\]

where Δx and Δp are the standard deviations (uncertainties) of position
and momentum, respectively, and ℏ is the reduced Planck constant. This
inequality is not a statement about experimental limitations or
measurement disturbance, but rather reflects a fundamental property
inherent in the mathematical structure of quantum mechanics itself.

\subsubsection{1.2 Gaussian Wave Packets as Minimum-Uncertainty
States}\label{gaussian-wave-packets-as-minimum-uncertainty-states}

Among all possible quantum states, Gaussian wave packets occupy a
privileged position: they saturate the uncertainty bound at the initial
time, achieving equality in the Heisenberg relation {[}2{]}. For a
Gaussian state at t=0:

\[\Delta x \cdot \Delta p = \frac{\hbar}{2}\]

This property designates Gaussian wave packets as minimum-uncertainty
states at the initial moment, representing the closest quantum analog to
classical particles in their simultaneous localization in both position
and momentum space. While related to harmonic oscillator coherent states
in their minimum-uncertainty property, free-space Gaussian packets do
not maintain this property under time evolution and should not be
conflated with coherent states in the strict sense {[}7{]}. Their dual
nature---exhibiting optimal localization in both position and momentum
space at t=0---makes them ideal subjects for both theoretical analysis
and pedagogical demonstration.

\subsubsection{1.3 Computational Approach and
Objectives}\label{computational-approach-and-objectives}

Numerical simulations provide direct, visual validation of abstract
quantum mechanical principles that often resist intuitive understanding.
This work employs a straightforward computational framework based on
Fast Fourier Transform (FFT) techniques to:

\begin{enumerate}
\def\labelenumi{\arabic{enumi}.}
\tightlist
\item
  Construct Gaussian wave packets with systematically varied position
  spreads
\item
  Compute momentum distributions via discrete Fourier transformation
\item
  Quantify position and momentum uncertainties numerically
\item
  Verify the Heisenberg uncertainty relation at the initial time across
  parameter space
\item
  Simulate free time evolution to demonstrate wave packet spreading
\end{enumerate}

The primary objectives are to provide quantitative numerical
verification of the uncertainty principle at the initial time, to
illustrate the reciprocal relationship between conjugate uncertainties,
and to offer a reproducible computational tool suitable for educational
purposes in quantum mechanics courses.

\begin{center}\rule{0.5\linewidth}{0.5pt}\end{center}

\subsection{2. Theory and Method}\label{theory-and-method}

\subsubsection{2.1 Quantum Mechanical
Framework}\label{quantum-mechanical-framework}

\paragraph{2.1.1 Wave Function and Probability
Interpretation}\label{wave-function-and-probability-interpretation}

In one-dimensional quantum mechanics, the state of a particle is
described by a complex-valued wave function ψ(x,t) satisfying the
Schrödinger equation. The probability density for finding the particle
at position x is:

\[\rho(x) = |\psi(x)|^2\]

normalized such that:

\[\int_{-\infty}^{\infty} |\psi(x)|^2 dx = 1\]

\paragraph{2.1.2 Position-Momentum
Duality}\label{position-momentum-duality}

The momentum-space representation is obtained through the Fourier
transform:

\[\Psi(p) = \frac{1}{\sqrt{2\pi\hbar}}\int_{-\infty}^{\infty} \psi(x)e^{-ipx/\hbar}dx\]

with corresponding probability density in momentum space:

\[\rho_p(p) = |\Psi(p)|^2\]

This dual representation---position and momentum as complementary
descriptions---lies at the heart of the uncertainty principle.

\subsubsection{2.2 Gaussian Wave Packet
Construction}\label{gaussian-wave-packet-construction}

\paragraph{2.2.1 Initial State}\label{initial-state}

A Gaussian wave packet centered at position x₀ with mean momentum p₀ has
the analytical form:

\[\psi(x,0) = \left(\frac{1}{2\pi\sigma_x^2}\right)^{1/4} \exp\left[-\frac{(x-x_0)^2}{4\sigma_x^2} + \frac{ip_0(x-x_0)}{\hbar}\right]\]

where σₓ is the initial position spread. For my simulations, I set x₀ =
0 and p₀ = 0 (particle at rest at the origin):

\[\psi(x,0) = \left(\frac{1}{2\pi\sigma_x^2}\right)^{1/4} \exp\left[-\frac{x^2}{4\sigma_x^2}\right]\]

\paragraph{2.2.2 Momentum Distribution}\label{momentum-distribution}

The Fourier transform of the initial Gaussian yields the momentum
distribution:

\[\Psi(p,0) = \left(\frac{2\sigma_x^2}{\pi\hbar^2}\right)^{1/4} \exp\left[-\frac{\sigma_x^2 p^2}{\hbar^2}\right]\]

This is also Gaussian, with spread:

\[\sigma_p = \frac{\hbar}{2\sigma_x}\]

demonstrating the reciprocal relationship: as σₓ increases, σₚ decreases
proportionally.

\subsubsection{2.3 Uncertainty
Quantification}\label{uncertainty-quantification}

\paragraph{2.3.1 Position Uncertainty}\label{position-uncertainty}

The position uncertainty is defined as the standard deviation:

\[\Delta x = \sqrt{\langle x^2 \rangle - \langle x \rangle^2}\]

where expectation values are computed as:

\[\langle x \rangle = \int_{-\infty}^{\infty} x|\psi(x)|^2 dx\]

\[\langle x^2 \rangle = \int_{-\infty}^{\infty} x^2|\psi(x)|^2 dx\]

For a Gaussian wave packet, Δx = σₓ exactly.

\paragraph{2.3.2 Momentum Uncertainty}\label{momentum-uncertainty}

Similarly, the momentum uncertainty is:

\[\Delta p = \sqrt{\langle p^2 \rangle - \langle p \rangle^2}\]

with:

\[\langle p \rangle = \int_{-\infty}^{\infty} p|\Psi(p)|^2 dp\]

\[\langle p^2 \rangle = \int_{-\infty}^{\infty} p^2|\Psi(p)|^2 dp\]

For the Gaussian momentum distribution, Δp = ℏ/(2σₓ).

\paragraph{2.3.3 Uncertainty Product at Initial
Time}\label{uncertainty-product-at-initial-time}

Combining these results for t=0:

\[\Delta x \cdot \Delta p = \sigma_x \cdot \frac{\hbar}{2\sigma_x} = \frac{\hbar}{2}\]

This confirms analytically that Gaussian wave packets achieve the
minimum uncertainty bound at the initial time.

\paragraph{2.3.4 The Generalized Uncertainty
Relation}\label{the-generalized-uncertainty-relation}

The standard Heisenberg relation Δx·Δp ≥ ℏ/2 is actually a special case
of the more general Schrödinger--Robertson uncertainty relation {[}3{]}:

\[\Delta x \cdot \Delta p \geq \sqrt{\left(\frac{\hbar}{2}\right)^2 + |\text{Cov}(x,p)|^2}\]

where the covariance is defined as:

\[\text{Cov}(x,p) = \frac{1}{2}\langle \hat{x}\hat{p} + \hat{p}\hat{x} \rangle - \langle \hat{x} \rangle \langle \hat{p} \rangle\]

For a Gaussian wave packet at t=0 with zero mean momentum (p₀=0),
Cov(x,p)=0 and the product saturates the minimal bound: Δx·Δp = ℏ/2.

\textbf{Time evolution:} During free propagation, the momentum-position
correlation develops (Cov(x,p) ≠ 0 for t\textgreater0), causing the
uncertainty product to exceed ℏ/2 even though the wavefunction remains
Gaussian. The numerical simulation presented here focuses on the initial
(t=0) minimum-uncertainty configuration.

\subsubsection{2.4 Time Evolution}\label{time-evolution}

\paragraph{2.4.1 Free Particle Dynamics}\label{free-particle-dynamics}

For a free particle (no external potential), the time-evolved wave
function can be obtained exactly. \textbf{For an initially
minimum-uncertainty Gaussian wave packet with σₓ(0) and zero initial
momentum (p₀=0)}, the position uncertainty evolves as:

\[\sigma_x(t) = \sigma_x(0)\sqrt{1 + \left(\frac{\hbar t}{2m\sigma_x^2(0)}\right)^2}\]

where m is the particle mass. This formula applies specifically to our
initial conditions; more general initial states would exhibit different
spreading behavior depending on their initial phase-space structure.
This demonstrates that position uncertainty grows monotonically with
time---a phenomenon known as wave packet spreading.

\paragraph{2.4.2 Momentum Conservation}\label{momentum-conservation}

Crucially, the momentum-space \textbf{probability density} remains
unchanged during free evolution:

\[|\Psi(p,t)|^2 = |\Psi(p,0)|^2\]

However, the momentum-space wavefunction itself acquires a
time-dependent phase:

\[\Psi(p,t) = \Psi(p,0) \cdot \exp\left(-i\frac{p^2 t}{2m\hbar}\right)\]

This phase evolution, while not affecting ⟨p⟩ or Δp, is physically
significant for interference phenomena and wave packet reconstruction.
Therefore:

\[\Delta p(t) = \Delta p(0) = \text{constant}\]

The uncertainty product thus evolves as:

\[\Delta x(t) \cdot \Delta p = \sigma_x(t) \cdot \frac{\hbar}{2\sigma_x(0)} > \frac{\hbar}{2} \quad \text{for } t>0\]

Note that for t\textgreater0, the product \textbf{strictly exceeds} ℏ/2
because x-p correlations develop during free evolution. The
minimum-uncertainty property (Δx·Δp = ℏ/2) holds only at t=0 for our
initial Gaussian state. The general Schrödinger--Robertson relation
remains satisfied throughout.

\subsubsection{2.5 Numerical
Implementation}\label{numerical-implementation}

\paragraph{2.5.1 Spatial Discretization}\label{spatial-discretization}

I discretize position space on a uniform grid with N = 16384 points:

\[x_j = -\frac{L}{2} + j\Delta x, \quad j = 0,1,\ldots,N-1\]

where L = 200 is the spatial domain size and Δx = L/N is the grid
spacing. The wave function is sampled at these points: ψⱼ = ψ(xⱼ).

\paragraph{2.5.2 Fourier Transform}\label{fourier-transform}

The momentum distribution is computed using the Fast Fourier Transform
(FFT):

\[\Psi_k = \text{FFT}[\psi_j] \cdot \frac{\Delta x}{\sqrt{2\pi\hbar}}\]

with corresponding momentum grid:

\[p_k = \hbar k_k, \quad k_k = \frac{2\pi}{L}\left(k - \frac{N}{2}\right)\]

for k = 0, 1, \ldots, N-1.

\paragraph{2.5.3 Expectation Value
Calculation}\label{expectation-value-calculation}

Position expectation values are computed numerically using trapezoidal
integration:

\[\langle x \rangle \approx \sum_{j=0}^{N-1} x_j |\psi_j|^2 \Delta x\]

\[\langle x^2 \rangle \approx \sum_{j=0}^{N-1} x_j^2 |\psi_j|^2 \Delta x\]

Similarly for momentum:

\[\langle p \rangle \approx \sum_{k=0}^{N-1} p_k |\Psi_k|^2 \Delta p\]

\[\langle p^2 \rangle \approx \sum_{k=0}^{N-1} p_k^2 |\Psi_k|^2 \Delta p\]

where Δp = 2πℏ/L is the momentum grid spacing.

\paragraph{2.5.4 Time Evolution
Algorithm}\label{time-evolution-algorithm}

Free time evolution is implemented using the split-operator method. The
kinetic energy operator in momentum space is:

\[\hat{K} = \frac{p^2}{2m}\]

For each time step dt:

\begin{enumerate}
\def\labelenumi{\arabic{enumi}.}
\tightlist
\item
  Transform to momentum space: Ψ(p) = FFT{[}ψ(x){]}
\item
  Apply kinetic evolution: Ψ(p) → Ψ(p) exp{[}-iKdt/(2ℏ){]}
\item
  Transform back: ψ(x) = IFFT{[}Ψ(p){]}
\item
  Repeat step 2 (second half-step for second-order accuracy)
\end{enumerate}

This algorithm is unitary and preserves norm to machine precision.

\begin{center}\rule{0.5\linewidth}{0.5pt}\end{center}

\subsection{3. System Configuration}\label{system-configuration}

\subsubsection{3.1 Simulation Parameters}\label{simulation-parameters}

The numerical simulations were conducted with the following parameters:

\textbf{Grid Configuration:} - Number of points: N = 2¹⁴ = 16384 -
Spatial domain: x ∈ {[}-100, 100{]} (atomic units) - Grid spacing: Δx ≈
0.012 - Momentum domain: p ∈ {[}-π/Δx, π/Δx{]}

\textbf{Initial Wave Packets:} - Center position: x₀ = 0 - Center
momentum: p₀ = 0 - Position spread range: σₓ ∈ {[}0.5, 8.0{]} - Number
of σₓ samples: 16 (geometrically spaced)

\textbf{Time Evolution:} - Particle mass: m = 1 (atomic units) - Time
step: dt = 0.002 - Total evolution time: t\_max = 1.0 - Number of steps:
500

\textbf{Unit System:} - Atomic units: ℏ = 1, m = 1 - All quantities
dimensionless

\subsubsection{3.2 Numerical Accuracy}\label{numerical-accuracy}

To ensure numerical reliability:

\begin{itemize}
\tightlist
\item
  Spatial domain (L = 200) is significantly larger than the maximum wave
  packet width, minimizing boundary effects
\item
  FFT resolution is sufficient to accurately capture momentum
  distributions
\item
  Wave function normalization is verified at each step:
  ∫\textbar ψ\textbar²dx ≈ 1.0 within 10⁻⁶
\item
  Time step satisfies stability criterion for the split-operator method
\end{itemize}

\begin{center}\rule{0.5\linewidth}{0.5pt}\end{center}

\subsection{4. Results and Discussion}\label{results-and-discussion}

This chapter presents the results of a numerical simulation of Gaussian
wave packets, with a particular focus on the quantitative and
qualitative verification of the Heisenberg uncertainty principle. The
analysis covers the investigation of initial states, the behavior of the
uncertainty product, and the time evolution of the system.

In the simulation, I examined the behavior of quantum mechanical wave
packets in one dimension. The initial states were described by Gaussian
functions, the standard deviation (σₓ) of which was systematically
varied.

\textbf{1. The Relationship Between Position and Momentum Uncertainty}

The \texttt{heisenberg\_scan.csv} dataset and the
\texttt{uncertainty\_product.png} plot generated from it clearly
demonstrate the inverse relationship between position uncertainty (Δx)
and momentum uncertainty (Δp). As I increased the standard deviation of
the initial wave packet in position space, i.e., the value of Δx, the
measured standard deviation in momentum representation, Δp, decreased
accordingly. This behavior stems from the fundamental property of the
Fourier transform, which connects the position and momentum space
representations.

\begin{figure}
\centering
\pandocbounded{\includegraphics[keepaspectratio,alt={Uncertainty Product}]{figs/uncertainty_product.png}}
\caption{Uncertainty Product}
\end{figure}

\textbf{Figure 1}: The value of the product Δx·Δp as a function of the
initial position's standard deviation, σₓ. The product is constant and
approximately 0.5, which, in atomic units (where ℏ=1), corresponds to
the theoretical minimum of ℏ/2.

The most significant result is that the product of these two quantities,
Δx·Δp, remained constant across the investigated range at t=0. Based on
Figure 1 and the \texttt{heisenberg\_scan.csv} data, the value of the
product consistently hovers around 0.5. In an atomic unit system (ℏ=1),
this corresponds precisely to the minimum value allowed by the
Heisenberg relation, Δx·Δp ≥ ℏ/2, at the initial time.

\textbf{2. Wave Packet Density Distributions}

To visualize the structure of the wave packets, Figures 2 and 3 show the
probability density distributions in position and momentum space for a
representative state with an initial standard deviation of σₓ = 3.0.
Both distributions, as theoretically expected, have a Gaussian shape. A
wider distribution in position space (larger Δx) results in a narrower
distribution in momentum space (smaller Δp), visually confirming the
inverse proportionality inherent in the uncertainty principle.

\begin{figure}
\centering
\pandocbounded{\includegraphics[keepaspectratio,alt={Position Density}]{figs/position_density_sigma3.0.png}}
\caption{Position Density}
\end{figure}

\textbf{Figure 2}: The probability density of the wave packet in
position space (\textbar ψ(x)\textbar²) for an initial state with a
standard deviation of σₓ = 3.0.

\begin{figure}
\centering
\pandocbounded{\includegraphics[keepaspectratio,alt={Momentum Density}]{figs/momentum_density_sigma3.0.png}}
\caption{Momentum Density}
\end{figure}

\textbf{Figure 3}: The probability density of the wave packet in
momentum space (\textbar Ψ(p)\textbar²) for an initial state with a
standard deviation of σₓ = 3.0.

\textbf{3. Time Evolution of the Wave Packet}

The simulation was also extended to investigate the free time evolution
of the wave packet (Figure 4). During this process, the position
uncertainty, Δx(t), monotonically increases over time. This phenomenon
is known as ``wave packet spreading'' and arises because the different
plane wave components making up the wave packet propagate at different
velocities. It is important to note that since no external force acts on
the particle, its momentum distribution---and thus its momentum
uncertainty Δp---remains constant in time. As a consequence, the
uncertainty product Δx(t)·Δp exceeds the minimum value ℏ/2 for
t\textgreater0 due to the development of position-momentum correlations.

\begin{figure}
\centering
\pandocbounded{\includegraphics[keepaspectratio,alt={Free Spreading}]{figs/free_spreading_sigma_x_t.png}}
\caption{Free Spreading}
\end{figure}

\textbf{Figure 4}: The change in position uncertainty Δx(t) over time
for a freely propagating Gaussian wave packet. The wave packet spreads
out in time, resulting in an increase in Δx.

\begin{center}\rule{0.5\linewidth}{0.5pt}\end{center}

\subsubsection{Discussion}\label{discussion}

The presented numerical results confirm and illustrate fundamental
concepts of quantum mechanics from several perspectives.

\textbf{Numerical Verification of the Heisenberg Uncertainty Principle}

The simulation clearly and quantitatively validates the Heisenberg
uncertainty principle (Δx·Δp ≥ ℏ/2) at the initial time. The data shows
that an unavoidable, inverse relationship exists between the
uncertainties of these two physical quantities. Their product at t=0
achieves the theoretical limit of ℏ/2, which is the defining property of
minimum-uncertainty states.

\textbf{The Gaussian Wave Packet as a Minimum Uncertainty State}

My results highlight the special role that Gaussian wave packets play in
quantum mechanics at the initial moment. The fact that their uncertainty
product Δx·Δp assumes the minimum possible value, ℏ/2, at t=0 means that
these states are \textbf{minimum-uncertainty wave packets} at that
instant. In other words, a Gaussian wave packet at t=0 describes the
``most classical-like'' state possible, where a particle's position and
momentum are simultaneously defined with the highest possible precision
allowed by quantum mechanics.

\textbf{Time Evolution of Uncertainty and Correlation Development}

The study of time evolution shows that although the position uncertainty
(Δx(t)) increases during free evolution, the momentum uncertainty (Δp)
remains constant. Consequently, the uncertainty product, Δx(t)·Δp, also
increases over time and strictly exceeds ℏ/2 for t\textgreater0. This is
in perfect agreement with the generalized Schrödinger--Robertson
relation, as position-momentum correlations develop during free
propagation. The minimum-uncertainty property is thus a feature of the
initial state only, not preserved during free evolution.

\textbf{Educational and Illustrative Value}

Finally, this simulation possesses outstanding educational and
demonstrative value. The underlying Python code (which forms the basis
of the simulation) is easy to run and reproduce, allowing students and
researchers to interactively explore one of the most important and least
intuitive principles of quantum mechanics. The visual results (plots and
density distributions) effectively aid in understanding these concepts,
bridging the gap between abstract mathematical formalism and physical
reality.

\begin{center}\rule{0.5\linewidth}{0.5pt}\end{center}

\subsection{5. Conclusions}\label{conclusions}

Through a simple but rigorous numerical experiment, I have demonstrated
the validity of the Heisenberg Uncertainty Principle using Gaussian wave
packets in one spatial dimension. The key findings of this study are:

\begin{enumerate}
\def\labelenumi{\arabic{enumi}.}
\item
  \textbf{Quantitative verification}: The uncertainty product Δx·Δp
  consistently equals ℏ/2 (0.5 in atomic units) at t=0 across all
  initial conditions, confirming that Gaussian wave packets saturate the
  Heisenberg bound and achieve minimum uncertainty at the initial time.
\item
  \textbf{Reciprocal relationship}: Position and momentum uncertainties
  exhibit the predicted inverse proportionality Δp = ℏ/(2Δx),
  demonstrated both numerically and visually through Fourier-transformed
  distributions.
\item
  \textbf{Wave packet spreading}: Free time evolution shows monotonic
  growth of position uncertainty Δx(t) while momentum uncertainty Δp
  remains constant, consistent with the analytical prediction for free
  Gaussian packets. The uncertainty product exceeds ℏ/2 for
  t\textgreater0 due to correlation development.
\item
  \textbf{Minimum uncertainty states at initial time}: The results
  confirm that Gaussian wave packets represent optimal quantum states at
  t=0, simultaneously achieving the best possible localization in both
  position and momentum space at that moment.
\item
  \textbf{Fundamental quantum limit}: The constant uncertainty product
  at the initial time establishes the Heisenberg principle as an
  intrinsic property of quantum states rather than a limitation of
  measurement technology.
\end{enumerate}

This work demonstrates that straightforward numerical simulations can
provide rigorous validation of fundamental quantum mechanical
principles. The methodology presented here---combining analytical
theory, FFT-based computation, and systematic parameter
variation---offers a template for exploring quantum mechanics in
educational settings.

The reproducible Python code and clear visualizations make this study a
valuable pedagogical resource. Students and researchers can directly
explore one of quantum mechanics' most profound principles, observing
the wave-particle duality and complementarity that lie at the heart of
the quantum world. The minimal complexity of the implementation (using
only standard NumPy and Matplotlib libraries) ensures accessibility
while maintaining scientific rigor.

Future extensions of this framework could investigate non-Gaussian wave
packets, explore the effects of external potentials on uncertainty
evolution, examine the role of position-momentum correlations in more
detail, or study multi-dimensional systems and angular momentum
uncertainties. The split-operator time evolution algorithm demonstrated
here can be readily adapted to more complex Hamiltonians, providing a
versatile tool for computational quantum mechanics.

\begin{center}\rule{0.5\linewidth}{0.5pt}\end{center}

\subsection{Acknowledgments}\label{acknowledgments}

The author thanks the open-source scientific Python community (NumPy,
Matplotlib) for providing the computational tools that enabled this
work. This research was conducted independently without external
funding.

\begin{center}\rule{0.5\linewidth}{0.5pt}\end{center}

\subsection{References}\label{references}

{[}1{]} Heisenberg, W. (1927). Über den anschaulichen Inhalt der
quantentheoretischen Kinematik und Mechanik. \emph{Zeitschrift für
Physik}, 43(3-4), 172-198.

{[}2{]} Kennard, E. H. (1927). Zur Quantenmechanik einfacher
Bewegungstypen. \emph{Zeitschrift für Physik}, 44(4-5), 326-352.

{[}3{]} Robertson, H. P. (1929). The uncertainty principle.
\emph{Physical Review}, 34(1), 163.

{[}4{]} Griffiths, D. J., \& Schroeter, D. F. (2018). \emph{Introduction
to Quantum Mechanics} (3rd ed.). Cambridge University Press.

{[}5{]} Sakurai, J. J., \& Napolitano, J. (2017). \emph{Modern Quantum
Mechanics} (2nd ed.). Cambridge University Press.

{[}6{]} Cohen-Tannoudji, C., Diu, B., \& Laloë, F. (2019). \emph{Quantum
Mechanics, Volume 1: Basic Concepts, Tools, and Applications} (2nd ed.).
Wiley-VCH.

{[}7{]} Glauber, R. J. (1963). Coherent and incoherent states of the
radiation field. \emph{Physical Review}, 131(6), 2766.

\begin{center}\rule{0.5\linewidth}{0.5pt}\end{center}

\subsection{Appendix A: Computational
Details}\label{appendix-a-computational-details}

\subsubsection{A.1 Software
Implementation}\label{a.1-software-implementation}

The simulation was implemented in Python 3.12+ using: - NumPy 1.21+ for
numerical arrays and FFT operations - Matplotlib 3.4+ for visualization
- Standard library modules for I/O and metadata management

The complete source code is available at: {[}GitHub repository URL{]}

\subsubsection{A.2 Algorithm Pseudocode}\label{a.2-algorithm-pseudocode}

\begin{Shaded}
\begin{Highlighting}[]
\CommentTok{\# Heisenberg Uncertainty Simulation}
\ImportTok{import}\NormalTok{ numpy }\ImportTok{as}\NormalTok{ np}
\ImportTok{from}\NormalTok{ numpy.fft }\ImportTok{import}\NormalTok{ fft, fftshift, ifft}

\CommentTok{\# Parameters}
\NormalTok{N }\OperatorTok{=} \DecValTok{16384}  \CommentTok{\# grid points}
\NormalTok{L }\OperatorTok{=} \FloatTok{200.0}  \CommentTok{\# spatial extent}
\NormalTok{hbar }\OperatorTok{=} \FloatTok{1.0}  \CommentTok{\# atomic units}

\CommentTok{\# Spatial grid}
\NormalTok{x }\OperatorTok{=}\NormalTok{ (np.arange(N) }\OperatorTok{{-}}\NormalTok{ N}\OperatorTok{//}\DecValTok{2}\NormalTok{) }\OperatorTok{*}\NormalTok{ (L}\OperatorTok{/}\NormalTok{N)}
\NormalTok{dx }\OperatorTok{=}\NormalTok{ L}\OperatorTok{/}\NormalTok{N}

\CommentTok{\# Momentum grid}
\NormalTok{k }\OperatorTok{=}\NormalTok{ (np.arange(N) }\OperatorTok{{-}}\NormalTok{ N}\OperatorTok{//}\DecValTok{2}\NormalTok{) }\OperatorTok{*}\NormalTok{ (}\DecValTok{2}\OperatorTok{*}\NormalTok{np.pi}\OperatorTok{/}\NormalTok{L)}
\NormalTok{p }\OperatorTok{=}\NormalTok{ hbar }\OperatorTok{*}\NormalTok{ k}

\CommentTok{\# Gaussian wave packet}
\KeywordTok{def}\NormalTok{ gaussian\_packet(x, sigma\_x):}
\NormalTok{    A }\OperatorTok{=}\NormalTok{ (}\DecValTok{1}\OperatorTok{/}\NormalTok{(}\DecValTok{2}\OperatorTok{*}\NormalTok{np.pi}\OperatorTok{*}\NormalTok{sigma\_x}\OperatorTok{**}\DecValTok{2}\NormalTok{))}\OperatorTok{**}\FloatTok{0.25}
    \ControlFlowTok{return}\NormalTok{ A }\OperatorTok{*}\NormalTok{ np.exp(}\OperatorTok{{-}}\NormalTok{x}\OperatorTok{**}\DecValTok{2}\OperatorTok{/}\NormalTok{(}\DecValTok{4}\OperatorTok{*}\NormalTok{sigma\_x}\OperatorTok{**}\DecValTok{2}\NormalTok{))}

\CommentTok{\# Uncertainty calculation}
\KeywordTok{def}\NormalTok{ compute\_uncertainty(x, density, dx):}
\NormalTok{    mean }\OperatorTok{=}\NormalTok{ np.}\BuiltInTok{sum}\NormalTok{(x }\OperatorTok{*}\NormalTok{ density) }\OperatorTok{*}\NormalTok{ dx}
\NormalTok{    mean\_sq }\OperatorTok{=}\NormalTok{ np.}\BuiltInTok{sum}\NormalTok{(x}\OperatorTok{**}\DecValTok{2} \OperatorTok{*}\NormalTok{ density) }\OperatorTok{*}\NormalTok{ dx}
    \ControlFlowTok{return}\NormalTok{ np.sqrt(mean\_sq }\OperatorTok{{-}}\NormalTok{ mean}\OperatorTok{**}\DecValTok{2}\NormalTok{)}

\CommentTok{\# Scan over sigma\_x values}
\ControlFlowTok{for}\NormalTok{ sigma\_x }\KeywordTok{in}\NormalTok{ np.geomspace(}\FloatTok{0.5}\NormalTok{, }\FloatTok{8.0}\NormalTok{, }\DecValTok{16}\NormalTok{):}
\NormalTok{    psi }\OperatorTok{=}\NormalTok{ gaussian\_packet(x, sigma\_x)}
    
    \CommentTok{\# Position uncertainty}
\NormalTok{    rho\_x }\OperatorTok{=}\NormalTok{ np.}\BuiltInTok{abs}\NormalTok{(psi)}\OperatorTok{**}\DecValTok{2}
\NormalTok{    Delta\_x }\OperatorTok{=}\NormalTok{ compute\_uncertainty(x, rho\_x, dx)}
    
    \CommentTok{\# Momentum uncertainty (via FFT)}
\NormalTok{    Psi }\OperatorTok{=}\NormalTok{ fftshift(fft(psi)) }\OperatorTok{*}\NormalTok{ dx}\OperatorTok{/}\NormalTok{np.sqrt(}\DecValTok{2}\OperatorTok{*}\NormalTok{np.pi)}
\NormalTok{    rho\_p }\OperatorTok{=}\NormalTok{ np.}\BuiltInTok{abs}\NormalTok{(Psi)}\OperatorTok{**}\DecValTok{2}
\NormalTok{    Delta\_p }\OperatorTok{=}\NormalTok{ compute\_uncertainty(p, rho\_p, }\DecValTok{2}\OperatorTok{*}\NormalTok{np.pi}\OperatorTok{/}\NormalTok{L)}
    
\NormalTok{    product }\OperatorTok{=}\NormalTok{ Delta\_x }\OperatorTok{*}\NormalTok{ Delta\_p}
    \BuiltInTok{print}\NormalTok{(}\SpecialStringTok{f"σₓ=}\SpecialCharTok{\{}\NormalTok{sigma\_x}\SpecialCharTok{:.2f\}}\SpecialStringTok{: Δx·Δp=}\SpecialCharTok{\{}\NormalTok{product}\SpecialCharTok{:.4f\}}\SpecialStringTok{"}\NormalTok{)}
\end{Highlighting}
\end{Shaded}

\subsubsection{A.3 Computational
Performance}\label{a.3-computational-performance}

\begin{itemize}
\tightlist
\item
  Single uncertainty calculation: \textasciitilde10 ms
\item
  Full σₓ scan (16 points): \textasciitilde160 ms
\item
  Time evolution (500 steps): \textasciitilde1.5 s
\item
  Memory usage: \textless200 MB
\item
  Platform: Google Colab (standard runtime)
\end{itemize}

The FFT-based algorithm scales as O(N log N), enabling efficient
computation even for large grid sizes.

\begin{center}\rule{0.5\linewidth}{0.5pt}\end{center}

\subsection{Appendix B: Data
Availability}\label{appendix-b-data-availability}

All simulation data, including: - Raw uncertainty values
(\texttt{heisenberg\_scan.csv}) - Position and momentum distributions
(NumPy arrays) - Time evolution data - High-resolution figures (PNG, 300
DPI) - Simulation metadata (\texttt{run\_info.txt})

are archived with DOI:
\href{https://doi.org/10.5281/zenodo.17390177}{10.5281/zenodo.17356922}
and available at the associated GitHub repository.

\begin{center}\rule{0.5\linewidth}{0.5pt}\end{center}

\textbf{Manuscript Version:} 2.0 (Revised)\\
\textbf{Word Count:} \textasciitilde5,600\\
\textbf{Figures:} 4\\
\textbf{Code Availability:} GitHub
\href{https://github.com/SteviLen420/Heisenberg_Uncertainty_Simulation/tree/main/Heisenberg_Uncertainty_pipeline}{repository
URL}\\
\textbf{Data Availability:} Zenodo DOI
\href{https://doi.org/10.5281/zenodo.17390177}{10.5281/zenodo.17356922}

\begin{center}\rule{0.5\linewidth}{0.5pt}\end{center}

\emph{Correspondence:} Stefan Len, tqe.simulation@gmail.com,
\href{https://github.com/SteviLen420/Heisenberg_Uncertainty_Simulation}{GitHub:
@SteviLen420}

\emph{Submitted to:} arXiv {[}quant-ph{]} or {[}physics.ed-ph{]}\\
\emph{Date:} October 20, 2025

\end{document}
