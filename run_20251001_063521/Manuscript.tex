\subsection{Results and Discussion}\label{results-and-discussion}

(run\_20251001\_063521)

This chapter presents the results of a numerical simulation of Gaussian
wave packets, with a particular focus on the quantitative and
qualitative verification of the Heisenberg uncertainty principle. The
analysis covers the investigation of initial states, the behavior of the
uncertainty product, and the time evolution of the system.

\subsubsection{Presentation of Results}\label{presentation-of-results}

In the simulation, we examined the behavior of quantum mechanical wave
packets in one dimension. The initial states were described by Gaussian
functions, the standard deviation (\(σ_x\)) of which was systematically
varied.

\textbf{1. The Relationship Between Position and Momentum Uncertainty}

The \texttt{heisenberg\_scan.csv} dataset and the
\texttt{uncertainty\_product.png} plot generated from it clearly
demonstrate the inverse relationship between position uncertainty
(\(Δx\)) and momentum uncertainty (\(Δp\)). As we increased the standard
deviation of the initial wave packet in position space, i.e., the value
of \(Δx\), the measured standard deviation in momentum representation,
\(Δp\), decreased accordingly. This behavior stems from the fundamental
property of the Fourier transform, which connects the position and
momentum space representations.

\textbf{Figure 1}: The value of the product \(Δx·Δp\) as a function of
the initial position's standard deviation, \(σ_x\). The product is
constant and approximately 0.5, which, in atomic units (where \(ħ=1\)),
corresponds to the theoretical minimum of \(ħ/2\) .

The most significant result is that the product of these two quantities,
\(Δx·Δp\), remained constant across the investigated range. Based on
Figure 1 and the \texttt{heisenberg\_scan.csv} data, the value of the
product consistently hovers around \(0.5\). In an atomic unit system
(\(ħ=1\)), this corresponds precisely to the minimum value allowed by
the Heisenberg relation, \(Δx·Δp ≥ ħ/2\).

\textbf{2. Wave Packet Density Distributions}

To visualize the structure of the wave packets, Figures 2 and 3 show the
probability density distributions in position and momentum space for a
representative state with an initial standard deviation of
\(σ_x = 3.0\). Both distributions, as theoretically expected, have a
Gaussian shape. A wider distribution in position space (larger \(Δx\))
results in a narrower distribution in momentum space (smaller \(Δp\)),
visually confirming the inverse proportionality inherent in the
uncertainty principle.

\textbf{Figure 2}: The probability density of the wave packet in
position space (\(|\psi(x)|^2\)) for an initial state with a standard
deviation of \(σ_x = 3.0\).

\textbf{Figure 3}: The probability density of the wave packet in
momentum space (\(|\phi(p)|^2\)) for an initial state with a standard
deviation of \(σ_x = 3.0\)

\textbf{3. Time Evolution of the Wave Packet}

The simulation was also extended to investigate the free time evolution
of the wave packet (Figure 4). During this process, the position
uncertainty, \(Δx(t)\), monotonically increases over time. This
phenomenon is known as ``wave packet spreading'' and arises because the
different plane wave components making up the wave packet propagate at
different velocities. It is important to note that since no external
force acts on the particle, its momentum distribution---and thus its
momentum uncertainty \(Δp\)---remains constant in time.

Figure 4: The change in position uncertainty \(Δx(t)\) over time for a
freely propagating Gaussian wave packet. The wave packet spreads out in
time, resulting in an increase in \(Δx\).

\begin{center}\rule{0.5\linewidth}{0.5pt}\end{center}

\subsubsection{Discussion}\label{discussion}

The presented numerical results confirm and illustrate fundamental
concepts of quantum mechanics from several perspectives.

\textbf{Numerical Verification of the Heisenberg Uncertainty Principle}
The simulation clearly and quantitatively validates the Heisenberg
uncertainty principle (\(Δx·Δp ≥ ħ/2\)). The data shows that an
unavoidable, inverse relationship exists between the uncertainties of
these two physical quantities. Their product never falls below the
theoretical limit of \(ħ/2\), which is an inherent property of quantum
systems.

\textbf{The Gaussian Wave Packet as a Minimum Uncertainty State} Our
results highlight the special role that Gaussian wave packets play in
quantum mechanics. The fact that their uncertainty product \(Δx·Δp\)
assumes the minimum possible value, \(ħ/2\), means that these states are
\textbf{minimum uncertainty wave packets}. In other words, a Gaussian
wave packet describes the ``most classical-like'' state possible, where
a particle's position and momentum are simultaneously defined with the
highest possible precision.

\textbf{Time Evolution of Uncertainty} The study of time evolution shows
that although the position uncertainty (\(Δx(t)\)) increases during free
evolution, the momentum uncertainty (\(Δp\)) remains constant.
Consequently, the uncertainty product, \(Δx(t)·Δp\), also increases over
time. This is in perfect agreement with the Heisenberg relation, as the
product continues to satisfy the inequality \(Δx(t)·Δp ≥ ħ/2\); it
simply moves away from the minimum value as time progresses.

\textbf{Educational and Illustrative Value} Finally, this simulation
possesses outstanding educational and demonstrative value. The
underlying Python code (which forms the basis of the simulation) is easy
to run and reproduce, allowing students and researchers to interactively
explore one of the most important and least intuitive principles of
quantum mechanics. The visual results (plots and density distributions)
effectively aid in understanding these concepts, bridging the gap
between abstract mathematical formalism and physical reality.
